\poemtitle{White-Eyes}

\begin{poem}
\begin{stanza}
\textsc{In} winter\verseline
~~~~all the singing is in\verseline
~~~~~~~~~the tops of the trees\verseline
~~~~~~~~~~~~~where the wind-bird
\end{stanza}

\begin{stanza}
with its white eyes\verseline
~~~~shoves and pushes\verseline
~~~~~~~~~among the branches.\verseline
~~~~~~~~~~~~~Like any of us
\end{stanza}

\begin{stanza}
he wants to go to sleep,\verseline
~~~~but he's restless--\verseline
~~~~~~~~~he has an idea,\verseline
~~~~~~~~~~~~~and slowly it unfolds
\end{stanza}

\begin{stanza}
from under his beating wings\verseline
~~~~as long as he stays awake.\verseline
~~~~~~~~~But his big, round music, after all,\verseline
~~~~~~~~~~~~~is too breathy to last.
\end{stanza}

\begin{stanza}
So, it's over.\verseline
~~~~In the pine-crown\verseline
~~~~~~~~~he makes his nest,\verseline
~~~~~~~~~~~~~he's done all he can.
\end{stanza}

\begin{stanza}
I don't know the name of this bird,\verseline
~~~~I only imagine his glittering beak\verseline
~~~~~~~~~tucked in a white wing\verseline
~~~~~~~~~~~~~while the clouds--
\end{stanza}

\begin{stanza}
which he has summoned\verseline
~~~~from the north--\verseline
~~~~~~~~~which he has taught\verseline
~~~~~~~~~~~~~to be mild, and silent--
\end{stanza}

\begin{stanza}
thicken, and begin to fall\verseline
~~~~into the world below\verseline
~~~~~~~~~like stars, or the feathers\verseline
~~~~~~~~~~~~~~~of some unimaginable bird
\end{stanza}

\begin{stanza}
that loves us,\verseline
~~~~that is asleep now, and silent--\verseline
~~~~~~~~~that has turned itself\verseline
~~~~~~~~~~~~~into snow.
\end{stanza}
\end{poem}