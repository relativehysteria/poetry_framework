\poemtitle{Death at a Great Distance}

\begin{poem}
\begin{stanza}
\textsc{The} ripe, floating caps\verseline
~~of the fly amanita\verseline
~~~~glow in the pinewoods.\verseline
~~~~~~I don't even think\verseline
~~~~~~~~of the eventual corruption of my body,
\end{stanza}

\begin{stanza}
but of how quaint and humorous they are,\verseline
~~like a collection of doorknobs,\verseline
~~~~half-moons,\verseline
~~~~~~then a yellow drizzle of flying saucers.\verseline
~~~~~~~~In any case
\end{stanza}

\begin{stanza}
they won't hurt me unless\verseline
~~I take them between my lips\verseline
~~~~and swallow, which I know enough\verseline
~~~~~~not to do. Once, in the south,\verseline
~~~~~~~~I had this happen:
\end{stanza}

\begin{stanza}
the soft rope of a water moccasin\verseline
~~slid down the red knees\verseline
~~~~of a mangrove, the hundreds of ribs\verseline
~~~~~~housed in their smooth, white\verseline
~~~~~~~~sleeves of muscle moving it
\end{stanza}

\begin{stanza}
like a happiness\verseline
~~toward the water, where some bubbles\verseline
~~~~on the surface of that underworld announced\verseline
~~~~~~a fatal carelessness. I didn't\verseline
~~~~~~~~even then move toward the fine point
\end{stanza}

\begin{stanza}
of the story, but stood in my lonely body\verseline
~~amazed and full of attention as it fell\verseline
~~~~like a stream of glowing syrup into\verseline
~~~~~~the dark water, as death\verseline
~~~~~~~~blurted out of that perfectly arranged mouth.
\end{stanza}
\end{poem}